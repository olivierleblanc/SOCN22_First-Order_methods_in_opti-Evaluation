%%%%%%%%%%%%%%%%%%%%%%%%%%%%%%%%%%%%%%%%%%
\section{Part 1 - Slide 45}
%%%%%%%%%%%%%%%%%%%%%%%%%%%%%%%%%%%%%%%%%%
%
%%%%%
\subsection{6. (with code)}
%
\begin{equation*}
    f(\bs x) = 2 x_{[1]} + x_{[2]} 
    = \max_{\bs y} \left\{ \sum_{i} y_i x_i; \sum_i y_i=3, 0 \le
    y_i \le 2 \right\} = \sigma_{\{ \bs y|\bs 1^\top \bs y = 3, 
    \bs 0 \le \bs y \le 2 \bs 1 \}}.
\end{equation*}
%
Hence,
%
\begin{equation*}
    \prox_f(\bs x) = \bs x - \cl P_{\{ \bs y|\bs 1^\top \bs y = 3, 
    \bs 0 \le \bs y \le 2 \bs 1 \}} (\bs x).
\end{equation*}
%
Writing $C=\{ \bs y|\bs 1^\top \bs y = 3, 
\bs 0 \le \bs y \le 2 \bs 1 \}$, it can be compared with 
$H_{\bs a, b} \bigcap \text{Box}[\bs l, \bs u]$, with 
$\bs a=\bs 1, b=3, \bs l=\bs 0, \bs u = \bs 2$.
%
\begin{python}
import numpy as np

def f(x):
    xsorted = np.sort(x)
    return 2*xsorted[0] + xsorted[1]

def projbox(x, l, u):
    return np.minimum(np.maximum(x,l), u)

def projH_inter_box(x, a, b, l, u):
    mu = 1
    factor = 1
    val = 10

    while (np.abs(val)>1e-8):
        val = a@projbox(x-mu*a, l, u) - b
        mu *= (1+factor)**(np.sign(val))
        factor /= 1.2

    return projbox(x-mu*a, l, u)

def proxf(x):
    return x - projH_inter_box(x, np.ones(len(x)), 3, np.zeros(len(x)), 2*np.ones(len(x)))  

print(proxf(np.array([2,1,4,1,2,1])))
\end{python}
%
The output is : $[1.49999994, 1., 2., 1., 1.49999994, 1.]$.
%
%%%%
\subsection{8.}
%
\begin{equation*}
    f(t) = \left\{
        \begin{array}{ll}
          1/t,~~t>0, \\
            \infty,~~ \text{else}.
        \end{array}
      \right.
\end{equation*}
%
\begin{equation*}
    \prox_{\lambda f}(t) = \argmin_u \left\{
        \begin{array}{ll}
          1/{\lambda u},~~u>0, \\
            \infty,~~ \text{else}.
        \end{array}
    \right. + \tinv 2 \norm{u-t}{2}^2 
\end{equation*}
%
Hence,
%
\begin{equation*}
    \frac{-1}{\lambda u^2} + u - t = 0 \Leftrightarrow 
    u^3 - \lambda tu^2-\lambda=0.
\end{equation*}
%
Finally,
%
\begin{equation}
    \prox_{\lambda f}(t) = \{ u>0 | u^3 
    - \lambda tu^2-\lambda=0  \}.
\end{equation}
%
%%%%
\subsection{9.}
%
\begin{equation*}
    f(\bs X) = \left\{
        \begin{array}{ll}
          \tr \bs X^{-1},~~\bs X \succ 0, \\
            \infty,~~ \text{else}.
        \end{array}
    \right.
    = \left\{
        \begin{array}{ll}
        \sum_{i=1}^n \tinv{\lambda_i},~~\bs X \succ 0, \\
            \infty,~~ \text{else}.
        \end{array}
    \right. 
\end{equation*}
%
As $\bs X \in \bb S^n$, it is a \emph{spectral function}.
Hence one can write $f(\bs X) = g(\lambda_1(\bs X), \cdots,
\lambda_n(\bs X)) = \sum_{i=1}^n h(\lambda_i)$. 
%
With the Singular Value Decomposition (SVD) of $\bs X$ as 
$\bs X = \bs U \diag{\lambda (\bs X)} \bs U^T$, this gives
%
\begin{align*}
\begin{split}
    \prox_{\lambda f} (\bs X) &= \bs U \diag(\prox_{\lambda g} 
    [\lambda_1,\cdots, \lambda_n]) \bs U^\top \\
    &= \bs U \diag(\prox_{\lambda h} (\lambda_1), \cdots,
    \prox_{\lambda h} (\lambda_n)) \bs U^\top \\
    &= \bs U \diag(\{ u>0| u^3-\lambda \lambda_i u^2-\lambda=0 \})
    \bs U^\top.
\end{split}
\end{align*}
%
%%%%%
\subsection{10.}
%
\begin{equation*}
    \lambda f(\bs x) = \lambda (\norm{\bs x}{2}-1)^2 
    = \lambda \norm{\bs x}{2}^2
    - 2 \lambda \norm{\bs x}{2} + \lambda.
\end{equation*}
%
Using the provided tables, one identifies it with 
$g(\bs x) + \frac{c}{2} \norm{\bs x}{2}^2 + \scp{\bs a}{\bs x}
+ \gamma$, with $g(\bs x) = -2\lambda\norm{\bs x}{2}, c=2\lambda,
\bs a=\bs 0, \gamma=0$.
Hence,
%
\begin{equation*}
    \prox_{\lambda f}(\bs x) = 
    \prox_{\frac{-2\lambda\norm{\cdot}{2}}{1+2\lambda}}
    \big(\frac{\bs x}{1+2\lambda} \big) = 
    \left( 1+ \frac{2\lambda}{\norm{\bs x}{2}} 
    \right) \frac{\bs x}{1+2\lambda},~~\bs x \neq \bs 0.
\end{equation*}
