%%%%%%%%%%%%%%%%%%%%%%%%%%%%%%%%%%%%%%%%%%
\section{Part 1 - Slide 45}
%%%%%%%%%%%%%%%%%%%%%%%%%%%%%%%%%%%%%%%%%%
%
%%%%%
\subsection{6. (with code)}
%
\begin{equation*}
    f(\bs x) = 2 x_{[1]} + x_{[2]} 
    = \max_{\bs y} \left\{ \sum_{i} y_i x_i; \sum_i y_i=3, 0 \le
    y_i \le 2 \right\} = \sigma_{\{ \bs y|\bs 1^\top \bs y = 3, 
    \bs 0 \le \bs y \le 2 \bs 1 \}}.
\end{equation*}
%
Hence,
%
\begin{equation*}
    \prox_f(\bs x) = \bs x - \cl P_{\{ \bs y|\bs 1^\top \bs y = 3, 
    \bs 0 \le \bs y \le 2 \bs 1 \}} (\bs x).
\end{equation*}
%
Writing $C=\{ \bs y|\bs 1^\top \bs y = 3, 
\bs 0 \le \bs y \le 2 \bs 1 \}$, it can be compared with 
$H_{\bs a, b} \bigcap \text{Box}[\bs l, \bs u]$, with 
$\bs a=\bs 1, b=3, \bs l=\bs 0, \bs u = \bs 2$.
%
\begin{python}
import numpy as np
def error_fct(a,b,l,u,x,mu):
    y = projbox(x-mu*a,l,u)
    return error = a@y-b
     
def projbox(x, l, u):
    return np.minimum(np.maximum(x,l), u)

def proj_H_inter_box(a,b,l,u,x):
    mu_low = -1 #start with guesses for mu-levels
    mu_high = 1
    #check that the levels give respectively negative and positive values
    #for the function error = a@y-1 with y = proj_box(l,u,x-mu*a)
    #positive for lower bound
    j=0
    j_max =100
    error_l = -1
    while (error_l<0) & (j<j_max) :
        mu_low = mu_low*2 #more negative
        error_l=error_fct(a,b,l,u,x,mu_low)
        j=j+1
    #negative for upper bound
    k=0
    k_max =10
    error_h = 1
    while (error_h>0) & (k<k_max) :
        mu_high = mu_high*2 #more positive 
        error_h=error_fct(a,b,l,u,x,mu_high)
        k=k+1

    i = 0
    i_max = 100
    tol = 1e-8
    error = 2*tol
    while (np.abs(error)>tol) & (i<i_max) :
        mu_mid = (mu_low+mu_high)/2
        error = error_fct(a,b,l,u,x,mu_mid)
        if error>0:
            mu_low = mu_mid
        else:
            mu_high = mu_mid
        i=i+1

    #Compute the solution with the good level
    return projbox(x-mu_mid*a,l,u)

def proxf(x):
    return x - proj_H_inter_box(np.ones(len(x)),3,np.zeros(len(x)),2*np.ones(len(x)),x)

x = np.array([2,1,4,1,2,1])
print(proxf(x))
\end{python}
%
The output is : $[1.5, 1., 2.,  1.,  1.5, 1. ]$.
%
%%%%
\subsection{8.}
%
\begin{equation*}
    f(t) = \left\{
        \begin{array}{ll}
          1/t,~~t>0, \\
            \infty,~~ \text{else}.
        \end{array}
      \right.
\end{equation*}
%
\begin{equation*}
    \prox_{\lambda f}(t) = \argmin_u \left\{
        \begin{array}{ll}
            \lambda/{ u},~~u>0, \\
            \infty,~~ \text{else}.
        \end{array}
    \right. + \tinv 2 \norm{u-t}{2}^2 
\end{equation*}
%
Clearly, the minimum occurs when $u>0$, i.e. on the differentiable part. Hence,
%
\begin{equation*}
    \frac{-\lambda}{u^2} + u - t = 0 \Leftrightarrow 
    u^3 - tu^2-\lambda=0,
\end{equation*}
and it can be checked the second derivative is always positive on $u>0$, implying it exists a unique solution of the above and it corresponds to a minimum. Finally,
%
\begin{equation}
    \prox_{\lambda f}(t) = \{ u>0 | u^3 
    - tu^2-\lambda=0  \}.
\end{equation}
%
%%%%
\subsection{9.}
%
\begin{equation*}
    f(\bs X) = \left\{
        \begin{array}{ll}
          \tr \bs X^{-1},~~\bs X \succ 0, \\
            \infty,~~ \text{else}.
        \end{array}
    \right.
    = \left\{
        \begin{array}{ll}
        \sum_{i=1}^n \tinv{\lambda_i},~~\bs X \succ 0, \\
            \infty,~~ \text{else}.
        \end{array}
    \right. 
\end{equation*}
%
As one can write $f(\bs X) = g(\lambda_1(\bs X), \cdots,
\lambda_n(\bs X)) = \sum_{i=1}^n h(\lambda_i)$, $f$ is a \emph{symmetric spectral function}.
%
With the EigenValue Decomposition (EVD) of $\bs X$ as 
$\bs X = \bs U \diag{\lambda (\bs X)} \bs U^T$, this gives
%
\begin{align*}
\begin{split}
    \prox_{\lambda f} (\bs X) &= \bs U \diag(\prox_{\lambda g} 
    [\lambda_1,\cdots, \lambda_n]) \bs U^\top \\
    &= \bs U \diag(\prox_{\lambda h} (\lambda_1), \cdots,
    \prox_{\lambda h} (\lambda_n)) \bs U^\top \\
    &= \bs U \diag(\{\{ u>0| u^3-\lambda_i u^2-\lambda=0 \}\}_{i=1}^n)
    \bs U^\top,
\end{split}
\end{align*}
with $h(t)=\frac{1}{t}$ for $t>0$. 
%
%%%%%
\subsection{10.}
%
\begin{equation*}
    \lambda f(\bs x) = \lambda (\norm{\bs x}{2}-1)^2 
    = \lambda \norm{\bs x}{2}^2
    - 2 \lambda \norm{\bs x}{2} + \lambda.
\end{equation*}
%
Using the provided tables, one identifies it with 
$g(\bs x) + \frac{c}{2} \norm{\bs x}{2}^2 + \scp{\bs a}{\bs x}
+ \gamma$, with $g(\bs x) = -2\lambda\norm{\bs x}{2}, c=2\lambda,
\bs a=\bs 0, \gamma=0$.
Hence,
%
\begin{equation*}
    \prox_{\lambda f}(\bs x) = 
    \prox_{\frac{-2\lambda\norm{\cdot}{2}}{1+2\lambda}}
    \bigg(\frac{\bs x}{1+2\lambda} \bigg) = 
    \left\{
        \begin{array}{ll}
            \left( 1+ \frac{2\lambda}{\norm{\bs x}{2}} 
            \right) \frac{\bs x}{1+2\lambda},~~\bs x \neq \bs 0, \\
            \{ \bs u: \norm{\bs u}{}= \frac{2\lambda}{1+2\lambda} \},~~ \bs x = \bs 0.
        \end{array}
      \right.
\end{equation*}
